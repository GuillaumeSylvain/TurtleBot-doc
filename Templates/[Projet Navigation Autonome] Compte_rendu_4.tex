\documentclass[a4wide,10pt]{article}

\usepackage[utf8]{inputenc}
\usepackage[french]{babel}
\usepackage[margin=1.2in]{geometry}
\usepackage{enumitem}
\setlist{noitemsep}
\usepackage{color}
\usepackage{array}
\usepackage{hyperref}
\usepackage[compact]{titlesec}
\titlespacing*{\section}{0pt}{6pt}{6pt}
\titlespacing*{\subsection}{0pt}{6pt}{6pt}

\begin{document}
\thispagestyle{empty}

\begin{center}
\LARGE \bfseries  Compte-rendu de réunion numéro 4 \\
\large \bfseries  Date : Jeudi 12 janvier 2017

\vspace{0.33cm}
\end{center}

\begin{center}
\begin{tabular}{ p{2.2cm}  p{13.6cm} }
\textbf{Objectifs} & Validation du premier prototype avec une démonstration et présentation de la seconde étape  \\
\textbf{Participants} & Michaël LAUER, Michel TAIX, Marine BOUCHET, Bruno DATO, Thibaut AGHNATIOS, Tristan KLEMPLA, Thibault LAGOUTE \\ 
\textbf{Auteur} & Thibault LAGOUTE \\
\end{tabular}
\end{center}

\section*{Sujets abordés}

Nous avons présenté un premier prototype et discuté des améliorations à ajouter avec les clients présents. De plus, ils nous ont conseillé et donné leur avis sur l'étape 2.


\section*{Démonstration}


Nous avons présenter le résultat du TP1 par une démonstration. Rappellons que le but de l'étape 1 est la détection d'une balle de couleur et l'avancement du robot jusqu'à une distance de 20 cm de cette dernière. \\  
	

\section*{Améliorations possibles du TP1}

Suite à l'expèrience, il faudrait ajouter ou modifier des fonctions pour le bon déroulement du TP1 et la sécurité du robot:

\begin{itemize}
\item amélioration de la détection de la balle avec la compacité
\item détection du sol
\item détection d'obstacle proche du Turtlebot(10cm)
\item arrêt du Turtlebot à 50cm de la balle
\item faire la commande en boucle fermée 
\item calcul des paramètres de l'odométrie
\end{itemize}

\section*{Mise en place Etape 2}

De plus, nous avons discuter de la plannification et du cahier des charges de l'étape 2 pour le déplacement d'amers(2D Qrcode). Le robot devra se déplacer d'amer en amer jusqu'à un amer final, dans une carte connue où il devra s'y localiser et détecter des obstacles statiques. Enfin, il devra générer une trajectoire idéale suivant un ou plusieurs critères, tout en essayant de la suivre avec une commande robuste. 


\section*{Décisions prises}

Nous avons décidé d'oublier l'état de l'Art pour nous consacrer essentiellement sur la réalisation du projet. Des tests, ainsi qu'un premier rapport sur l'odométrie seront réalisés pour la prochaine réunion.


\subsection*{\color{red}{Prochaine réunion : Vendredi 20 janvier 2017 (14h-15h)}}

\end{document}