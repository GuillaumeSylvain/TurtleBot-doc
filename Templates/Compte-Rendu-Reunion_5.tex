\documentclass[a4wide,10pt]{article}

\usepackage[utf8]{inputenc}
\usepackage[french]{babel}
\usepackage[margin=1in]{geometry}
\usepackage{enumitem}
\setlist{noitemsep}
\usepackage{color}
\usepackage{array}
\usepackage{hyperref}
\usepackage[compact]{titlesec}
\titlespacing*{\section}{0pt}{6pt}{6pt}
\titlespacing*{\subsection}{0pt}{6pt}{6pt}

\begin{document}
\thispagestyle{empty}

\begin{center}
\LARGE \bfseries  Compte-rendu de réunion numéro 5 \\
\large \bfseries  Date : Vendredi 20 janvier 2017
\vspace{0.05cm}
\end{center}

\begin{center}
\begin{tabular}{ p{2.2cm}  p{13.6cm} }
\textbf{Objectif(s)} & Bilan fin du bloc 2 de projet + discussion etape 3 sur les amers 3D  \\
\textbf{Participants} & Michel TAIX, Frederic LERASLE, Bruno DATO, Thibaut AGHNATIOS, Tristan KLEMPLA, Thibault LAGOUTE  \\ 
\textbf{Auteur} & Thibaut AGHNATIOS  \\
\end{tabular}
\end{center}

\section*{Sujets abordés}

La réunion avec le client s'est faite d'abord avec M. LERASLE de 12h à 12h30 sur la partie vision, détection d'amers et caméras. Ensuite, de 14h15 à 14h45, avec M. TAIX pour faire le point sur l'étape 2.

\subsection*{Bilan étape 2}

Lors de cette réunion, nous avons expliqué notre avancement sur la détection des amers 2D et la navigation d'amer en amer. Au niveau de la localisation, nous avons bien la détection des 
Qrcode qui permet de se localiser, cependant il est nécessaire d'être proche du Qrcode. Au niveau de la commande haut niveau, on utilise la navigation sous ROS, il reste encore les liens à faire entre la localisation et la génération de trajectoire. Ensuite au niveau de la trajectoire, la génération est correcte mais il faut l'optimiser par la suite avec l'odométrie. Au niveau de la commande, il faut faire les réglages pour que le suivi de trajectoire se fasse au mieux.\\


\subsection*{Les améliorations à faire pour les étapes 2 et 3}

M. LERASLE nous a dit d'utiliser au maximum 10 Qrcode avec des patterns simples pour une meilleure fiabilité de détection et une meilleure robustesse.  Ensuite mettre en place des zones de visibilité sur notre carte pour faciliter l'approche vers le Qrcode pour la détection. Ce point rejoint la proposition de M. TAIX qui consiste à la mise en place d'un graphe de Qrcode ou bien de potentiels pour faciliter la détection des différents amers. Au niveau de la caméra, il faudra faire la calibration pour de meilleures performances. Sur la vision toujours, il faudrait aussi calculer les différents degrés de confiance pour la localisation selon l'angle de vision du Qrcode. Pour l'étape 3 à propos de la détection des amers 3D, il faudra s'intéresser à l'algorithme "tabletop".\\

Suite au bilan de l'étape 2 avec M. TAIX, il est nécessaire de faire différents tests de performance de "visp" et modifier la taille des Qrcode pour augmenter la distance de détection. Au niveau de la commande haut niveau, il faut gérer la recherche des Qrcode en cherchant à une meilleure accroche et la mise en place d'un graphe pour le déplacement d'amer en amer. Au niveau de l'odométrie, il faut faire les tests et un rapport complet pour corriger l'erreur du Turtlebot. Au niveau de la planification des trajectoires, il faut prendre en compte l'évitement d'obstacles et vérifier la fiabilité de l'odométrie.\\

En fin de réunion, nous lui avons demandé de faire évoluer le cahier des charges en tenant compte que le planning s'est raccourci d'une semaine et qu'il faut aussi prendre en compte les différents livrables et présentations à finaliser lors de la dernière semaine. Une réunion est prévue pendant la phase de cours du troisième bloc pour nous donner un nouveau cahier des charges pour notre dernier bloc de projet.

\section*{Décision(s) prise(s)}

Nous n'avons pas la possibilité de travailler à l'AIP pendant le bloc de cours, cependant nous allons nous organiser de manière à être efficace en prévoyant les différentes tâches à effectuer lors du dernier bloc de projet. Une réunion va être programmer avec le client.

\subsection*{\color{red}{Prochaine réunion : dans 3 semaines environ pendant le bloc 3 de cours}}

\end{document}